<+	+>	!comp!	!exe!
%        File: !comp!expand("%")!comp!
%     Created: !comp!strftime("%a %b %d %I:00 %p %Y ").substitute(strftime('%Z'), '\<\(\w\)\(\w*\)\>\(\W\|$\)', '\1', 'g')!comp!
% Last Change: !comp!strftime("%a %b %d %I:00 %p %Y ").substitute(strftime('%Z'), '\<\(\w\)\(\w*\)\>\(\W\|$\)', '\1', 'g')!comp!
%
\documentclass[german,12pt]{article}

% Mathepaket, Deutsch, besseres Verbatim, einstellbare Listen,
% formatierbares Verbatim :

\usepackage{amsmath,german,moreverb,enumerate,alltt,url}
\usepackage[latin1]{inputenc} % latin1 input
\usepackage{color}            % einfachere Farben
\usepackage{graphicx}         % \includegraphics
\usepackage{fancyhdr}         % Kopf- und Fu"szeilen
\usepackage{tabularx}
\usepackage{babel}
\usepackage{subfigure}
\headheight 70.0000pt

\usepackage[style=long,border=none,header=plain,cols=3]{glossary}
\makeglossary
\renewcommand{\entryname}{K\"urzel}
\renewcommand{\descriptionname}{Beschreibung}

% hier Titel anpassen, dieser steht dann auf jeder Seite <FancyHeaders>
\newcommand{\thetitle}{
\LARGE{\textbf{ <+main topic+>}} \\
\large \textbf{<+sub topic+>} \\
\normalsize <+author(s)+>}

\newcommand{\theauthor}{<+author+>}
% addon TODO box .. zu verwenden mit \TODO{blablupp}
\newcommand{\TODO}[1]{\colorbox{red}{{\textbf{TODO: }}{#1}}}

\begin{document}

\begin{titlepage}
\center
\Large <+titel+>\\
\vspace{2em}
\Huge
\sffamily
\textbf{<+topic+>}\\
\vspace{2em}
\Large
<+author(s)+>\\
\vspace{2em}
<+date+>\\

% hier das Logo angeben (hier die Athene im unterordner pics/)
%\begin{center}
%  \includegraphics[scale = 0.50]{pics/athene_sw.jpg}
%\end{center}
%\large
Technische Universit\"at Darmstadt\\
Fachbereich Informatik\\
\vspace{1em}
<+dozent+>\\
\end{titlepage}

\lhead[\thetitle]{\theauthor}
\rhead[\theauthor]{\thetitle}
%\cfoot{\thepage}

%\renewcommand\thesection{Aufgabe~\arabic{section}}
%\renewcommand\thesubsection{\arabic{section}.\arabic{subsection}}

%\makeglossary

\newpage
%Inhaltsverzeichnis
\tableofcontents
\newpage

\pagestyle{fancy}
\pagenumbering{arabic}
%hier sections anfuegen im stile von
%\section{Einleitung}
%\input{inhalt/01einleitung/einleitung.tex}
%\newpage
\section{Glossar}
%\input{inhalt/glossar.tex}
\newpage
%---------------------------------------------------------------------------
\section{Anhang}
%\input{inhalt/anhang.tex}
\newpage
%---------------------------------------------------------------------------


%\glossary{name=set,description=a collection of objects}

% Dieses Command zum Erstellen des Glossars ausf�hren und danach nochmal kompilieren :
% makeindex -t pflichtenheft.glg -o pflichtenheft.gls -s pflichtenheft.ist pflichtenheft.glo
% nach
% http://theoval.sys.uea.ac.uk/~nlct/latex/thesis/node25.html

%\printglossary

\end{document}
